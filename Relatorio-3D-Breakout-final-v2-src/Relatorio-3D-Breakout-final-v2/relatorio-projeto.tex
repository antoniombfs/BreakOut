\documentclass[12pt,a4paper]{memoir}

\usepackage[portuguese]{babel}
\usepackage[T1]{fontenc}

\usepackage{makeidx}
\usepackage{xspace}
\usepackage{graphicx,color,times}
\usepackage{fancyhdr}
\usepackage{amsmath}
\usepackage{latexsym}
\usepackage[printonlyused]{acronym}
\usepackage{float}
\usepackage{listings}
\usepackage{tocbibind}
\usepackage[backend=biber,style=numeric,sorting=none]{biblatex}
\addbibresource{bibliografia.bib}

\usepackage{hyperref}

% Para procurar imagens também na pasta do relatório
\graphicspath{{./}}

\pagestyle{fancy}
\renewcommand{\chaptermark}[1]{\markboth{#1}{}}
\renewcommand{\sectionmark}[1]{\markright{\thesection\ #1}}
\fancyhf{} \fancyhead[LE,RO]{\bfseries\thepage}
\fancyhead[LO]{\bfseries\rightmark}
\fancyhead[RE]{\bfseries\leftmark}
\renewcommand{\headrulewidth}{0.5pt}
\renewcommand{\footrulewidth}{0pt}
\setlength{\headheight}{15.6pt}
\setlength{\marginparsep}{0cm}
\setlength{\marginparwidth}{0cm}
\setlength{\marginparpush}{0cm}
\addtolength{\hoffset}{-1.0cm}
\addtolength{\oddsidemargin}{\evensidemargin}
\addtolength{\oddsidemargin}{0.5cm}
\addtolength{\evensidemargin}{-0.5cm}

\usepackage{fix-cm}
\usepackage{fourier}
\usepackage[scaled=.92]{helvet}
\definecolor{ChapGrey}{rgb}{0.6,0.6,0.6}
\newcommand{\LargeFont}{
  \usefont{\encodingdefault}{\rmdefault}{b}{n}
  \fontsize{60}{80}\selectfont\color{ChapGrey}
  }
\makeatletter
\makechapterstyle{GreyNum}{
  \renewcommand{\chapnamefont}{\large\sffamily\bfseries\itshape}
  \renewcommand{\chapnumfont}{\LargeFont}
  \renewcommand{\chaptitlefont}{\Huge\sffamily\bfseries\itshape}
  \setlength{\beforechapskip}{0pt}
  \setlength{\midchapskip}{40pt}
  \setlength{\afterchapskip}{60pt}
  \renewcommand\chapterheadstart{\vspace*{\beforechapskip}}
  \renewcommand\printchaptername{
  \begin{tabular}{@{}c@{}}
    \chapnamefont \@chapapp\\}
    \renewcommand\chapternamenum{\noalign{\vskip 2ex}}
    \renewcommand\printchapternum{\chapnumfont\thechapter\par}
    \renewcommand\afterchapternum{
  \end{tabular}
  \par\nobreak\vskip\midchapskip}
  \renewcommand\printchapternonum{}
  \renewcommand\printchaptertitle[1]{
  {\chaptitlefont{##1}\par}}
  \renewcommand\afterchaptertitle{\par\nobreak\vskip \afterchapskip}
}
\makeatother
\chapterstyle{GreyNum}

\setcounter{tocdepth}{3}
\setsecnumdepth{subsubsection}

\renewcommand{\ttdefault}{lmtt}

% Listagens (C++)
\definecolor{dkgreen}{rgb}{0,0.6,0}
\definecolor{mauve}{rgb}{0.58,0,0.82}
\definecolor{gray}{rgb}{0.5,0.5,0.5}

\lstset{ %
  language=C++,
  basicstyle=\footnotesize\ttfamily,
  numbers=left,
  numberstyle=\tiny\color{gray},
  stepnumber=1,
  numbersep=6pt,
  showspaces=false,
  showstringspaces=false,
  showtabs=false,
  frame=single,
  tabsize=2,
  captionpos=b,
  breaklines=true,
  breakatwhitespace=false,
  keywordstyle=\color{blue},
  commentstyle=\color{dkgreen},
  stringstyle=\color{mauve},
  belowskip=0.2cm
}

\renewcommand{\lstlistingname}{Excerto de Código}
\renewcommand{\lstlistlistingname}{Lista de Excertos de Código}

\renewcommand{\today}{\day \ifcase \month \or Janeiro\or Fevereiro\or Março\or %
Abril\or Maio\or Junho\or Julho\or Agosto\or Setembro\or Outubro\or Novembro\or %
Dezembro\fi de \number \year}

\begin{document}

\thispagestyle{empty}
\setcounter{page}{-1}

\begin{center}
\begin{Huge}
\textbf{Universidade da Beira Interior}
\end{Huge}
\end{center}

\begin{center}
\begin{Huge}
Departamento de Informática
\end{Huge}
\end{center}

\vspace{0,07cm}
\begin{figure}[!htb]
\centering
\includegraphics[width=191pt]{ubi-fe-di.png}
\end{figure}

\vspace{0.6cm}
\begin{center}
\begin{Large}
\textbf{Relatório de Projeto: \emph{3D Breakout}}
\end{Large}
\end{center}

\vspace{0.2cm}
\begin{center}
\begin{normalsize}
Implementação de um jogo \emph{Breakout} em 3D com OpenGL (\textit{C++})
\end{normalsize}
\end{center}

\vspace{0.6cm}
\begin{center}
\begin{large}
Elaborado por:
\end{large}
\end{center}

\vspace{0.2cm}
\begin{center}
\begin{large}
\textbf{António Sobreiro (47933)}\\
\textbf{Leonardo Ferreira (52981)}
\end{large}
\end{center}

\vspace{0.35cm}
\begin{center}
\begin{large}
Unidade Curricular: \textbf{Computação Gráfica}
\end{large}
\end{center}

\vspace{0.35cm}
\begin{center}
\begin{large}
Docente/Orientador:
\end{large}
\end{center}

\vspace{0.2cm}
\begin{center}
\begin{large}
\textbf{Abel Gomes}\\
\textbf{João Dias}
\end{large}
\end{center}

\vspace{0.65cm}
\begin{center}
\begin{normalsize}
\today
\end{normalsize}
\end{center}


\clearpage{\thispagestyle{empty}\cleardoublepage}

\frontmatter

\chapter*{Agradecimentos}

Este trabalho foi desenvolvido no âmbito da unidade curricular \textbf{Computação Gráfica} do Departamento de Informática da Universidade da Beira Interior.

Agradecemos aos docentes \textbf{Abel Gomes} e \textbf{João Dias} pelo acompanhamento e pelas orientações ao longo do desenvolvimento do projeto, bem como pelo contexto prático da unidade curricular, que ajudou a consolidar conceitos de \ac{OpenGL} e programação gráfica.

Por fim, agradecemos também aos colegas com quem fomos discutindo ideias e soluções durante o semestre, o que acabou por ser útil sobretudo na fase de depuração e afinação do jogo.

\vspace{0.4cm}
\begin{flushright}
António Sobreiro (47933)\\
Leonardo Ferreira (52981)
\end{flushright}


\clearpage{\thispagestyle{empty}\cleardoublepage}

\tableofcontents
\clearpage{\thispagestyle{empty}\cleardoublepage}

\listoffigures
\clearpage{\thispagestyle{empty}\cleardoublepage}

\listoftables
\clearpage{\thispagestyle{empty}\cleardoublepage}

\lstlistoflistings
\clearpage{\thispagestyle{empty}\cleardoublepage}

\chapter*{Acrónimos}
\label{chap:acro}

\begin{acronym}[OpenGL]
  \acro{AABB}{\emph{Axis-Aligned Bounding Box}}
  \acro{API}{\emph{Application Programming Interface}}
  \acro{CPU}{\emph{Central Processing Unit}}
  \acro{FPS}{\emph{Frames Per Second}}
  \acro{GLFW}{\emph{Graphics Library Framework}}
  \acro{GLEW}{\emph{OpenGL Extension Wrangler}}
  \acro{GLM}{\emph{OpenGL Mathematics}}
  \acro{GLSL}{\emph{OpenGL Shading Language}}
  \acro{GPU}{\emph{Graphics Processing Unit}}
  \acro{HUD}{\emph{Heads-Up Display}}
  \acro{NDC}{\emph{Normalized Device Coordinates}}
  \acro{OpenGL}{\emph{Open Graphics Library}}
  \acro{UBI}{Universidade da Beira Interior}
  \acro{VAO}{\emph{Vertex Array Object}}
  \acro{VBO}{\emph{Vertex Buffer Object}}
\end{acronym}

\clearpage{\thispagestyle{empty}\cleardoublepage}

\mainmatter
\acresetall

\chapter{Introdução}
\label{chap:intro}

\section{Enquadramento}
\label{sec:enq}

O jogo \emph{Breakout} é um clássico: uma bola, uma raquete (paddle) e um conjunto de blocos que devem ser destruídos com sucessivas colisões. Apesar de ser um problema simples do ponto de vista de regras, é um ótimo pretexto para aplicar conceitos fundamentais de computação gráfica e desenvolvimento de jogos: ciclo principal (\emph{game loop}), transformação de objetos em 3D, iluminação e sombreamento, deteção de colisões e gestão de estado de jogo.

Este projeto consiste na implementação de uma versão \textbf{3D} do \emph{Breakout} usando \ac{OpenGL} (perfil \emph{core} 3.3) em \textbf{C++17}. O objetivo não foi criar um jogo comercial, mas sim construir uma base sólida e compreensível, com um código organizado por módulos e decisões técnicas justificadas.

O trabalho foi desenvolvido em grupo pelos alunos \textbf{António Sobreiro (47933)} e \textbf{Leonardo Ferreira (52981)}. Ao longo do desenvolvimento procurou-se manter uma divisão clara de tarefas, mas sobretudo uma integração contínua: mudanças pequenas e frequentes, com testes manuais imediatos para validar colisões, controlos e estabilidade do \emph{game loop}.

\section{Breve História do \emph{Breakout}}
\label{sec:hist}

O \emph{Breakout} nasceu na década de 70, numa altura em que os jogos de arcada eram limitados em recursos mas ricos em ideias. O conceito base --- uma bola que ressalta e uma ``barra'' controlada pelo jogador --- foi popularizado pelo \emph{Breakout} da Atari (1976)~\cite{breakoutAtari}. O sucesso do jogo veio muito da simplicidade: regras fáceis de entender em poucos segundos, mas com espaço para melhorar a destreza e a estratégia (por exemplo, abrir ``túneis'' nas paredes de blocos para a bola ficar a limpar a parte de cima).

Com o tempo, surgiram várias releituras do mesmo estilo de jogo (nomeadamente com diferentes tipos de blocos, bónus e níveis), o que o tornou um ponto de partida quase ``clássico'' em cursos de computação gráfica e desenvolvimento de jogos: é pequeno, mas obriga a juntar matemática, renderização e lógica de jogo num sistema coerente.

\section{Motivação}
\label{sec:mot}

A motivação principal para escolher este projeto foi dupla. Por um lado, é um jogo suficientemente pequeno para ser feito num contexto académico, mas ao mesmo tempo obriga-nos a passar por quase todos os pontos críticos de um \emph{engine} simples: input, atualização por \textit{delta time}, física básica e renderização. Por outro lado, sendo um projeto em 3D, permite aplicar com clareza o pipeline de transformações (modelo, vista e projeção) e perceber melhor a relação entre matemática e o resultado visual.

Além disso, este tipo de jogo é excelente para experimentar afinações (\emph{tuning}): velocidade da bola, sensibilidade do \emph{spin}, tamanhos do mundo, dificuldade e feedback visual (cores, paredes e linha de derrota).

\section{Objetivos}
\label{sec:obj}

Os objetivos definidos para o projeto foram:

\begin{itemize}
  \item Implementar um \emph{Breakout} em 3D com ciclo de jogo estável e atualização baseada em \textit{delta time};
  \item Controlar uma raquete no eixo $x$ com limites laterais e colisão com a bola;
  \item Gerar e desenhar um conjunto de blocos organizados em grelha, com destruição e pontuação;
  \item Implementar deteção de colisões entre esfera e \ac{AABB} (blocos/raquete) e resposta simples (reflexão em $x$ ou $y$);
  \item Renderizar a cena em 3D com iluminação do tipo Phong (ambiente + difusa + especular) em \ac{GLSL};
  \item Disponibilizar um processo de compilação e execução (Makefile) e documentar decisões, limitações e possíveis melhorias.
\end{itemize}

\section{Organização do Documento}
\label{sec:organ}

O relatório encontra-se organizado da seguinte forma:

\begin{enumerate}
  \item No Capítulo~\ref{chap:intro} apresenta-se o projeto, a motivação e os objetivos.
  \item No Capítulo~\ref{chap:tec} descrevem-se as tecnologias utilizadas e os conceitos base (pipeline de renderização, matrizes e iluminação).
  \item No Capítulo~\ref{chap:arquitetura} detalha-se a análise do problema, os requisitos e a arquitetura do sistema (classes, estados e ciclo de jogo).
  \item No Capítulo~\ref{chap:implementacao} descreve-se a implementação, incluindo os algoritmos principais (geração de geometria, colisões e renderização) e uma bateria de testes práticos.
  \item Por fim, no Capítulo~\ref{chap:conclusoes} apresentam-se conclusões e trabalho futuro.
  \item Os Apêndices~A e B incluem um manual de utilização (controlos) e notas de compilação/estrutura do projeto.
\end{enumerate}

\clearpage{\thispagestyle{empty}\cleardoublepage}

\chapter{Tecnologias e Conceitos Base}
\label{chap:tec}

\section{Introdução}
\label{chap2:sec:intro}

Este capítulo descreve as tecnologias e bibliotecas utilizadas no projeto, bem como os conceitos que suportam as decisões tomadas ao longo do desenvolvimento. Apesar de ser um jogo relativamente simples, ele usa a ``cadeia completa'' de uma aplicação gráfica em tempo real: criação de janela e contexto \ac{OpenGL}, compilação de shaders, gestão de buffers, câmaras e transformações, e um modelo básico de iluminação.

\section{Linguagem e Ambiente}
\label{chap2:sec:ambiente}

O projeto foi desenvolvido em \textbf{C++17}, com foco num código curto e direto, mas sem misturar responsabilidades. A estrutura do projeto segue uma organização típica por diretórios (\texttt{src/}, \texttt{include/}, \texttt{shaders/}) e o processo de build é suportado por um \texttt{Makefile} (ver Apêndice~B).

Apesar de o \texttt{Makefile} estar configurado com bibliotecas de \textbf{Windows} (por exemplo, \texttt{-lopengl32}), o código é suficientemente portátil para que, com pequenas alterações nos \textit{flags} de ligação, possa ser compilado em Linux/macOS.

\section{\ac{OpenGL} 3.3 e Pipeline de Renderização}
\label{chap2:sec:opengl}

O \ac{OpenGL} é uma \ac{API} de renderização em tempo real onde o programador controla explicitamente o envio de dados para a \ac{GPU} e o modo como a geometria é processada. Neste projeto é usado o perfil \emph{core} 3.3, o que implica que toda a geometria é desenhada através de \ac{VAO} e \ac{VBO}, e que o pipeline programável é definido em \ac{GLSL} (vertex e fragment shaders)~\cite{openglSpec,glslSpec}.

O fluxo típico por frame é:
\begin{enumerate}
  \item Ativar o shader (programa);
  \item Atualizar \emph{uniforms} (matrizes, luz, cor do objeto);
  \item Selecionar o \ac{VAO} correspondente (cubo ou esfera);
  \item Chamar \texttt{glDrawArrays} para desenhar.
\end{enumerate}

\section{GLFW, GLEW e GLM}
\label{chap2:sec:bibs}

Foram usadas três bibliotecas essenciais:

\begin{itemize}
  \item \textbf{\ac{GLFW}}: criação de janela, contexto \ac{OpenGL} e captura de eventos de teclado. Permite também gerir o \emph{swap} de buffers e o \emph{polling} de eventos~\cite{glfw}.
  \item \textbf{\ac{GLEW}}: carregamento de extensões e funções \ac{OpenGL}, necessário em muitos ambientes para aceder à \ac{API} moderna~\cite{glew}.
  \item \textbf{\ac{GLM}}: biblioteca de matemática que replica a sintaxe do GLSL (vetores e matrizes) e simplifica operações como \texttt{lookAt} e \texttt{perspective}~\cite{glm}.
\end{itemize}

\section{Transformações e Câmara (Model--View--Projection)}
\label{chap2:sec:mvp}

A renderização 3D baseia-se no uso de três matrizes principais:

\begin{itemize}
  \item \textbf{Model}: posiciona e escala cada objeto no mundo;
  \item \textbf{View}: representa a câmara (onde estamos e para onde olhamos);
  \item \textbf{Projection}: define a perspetiva (ângulo de visão e plano de corte).
\end{itemize}

No projeto, a câmara é fixa, posicionada em:
\[
\texttt{cameraPos} = (0, 0, 25)
\]
e a projeção é perspetiva com campo de visão de $45^\circ$ e planos $0.1$ e $100.0$:

\begin{lstlisting}[caption={Configuração da câmara e projeção.},label={lst:camera}]
view = glm::lookAt(cameraPos, cameraTarget, cameraUp);
projection = glm::perspective(glm::radians(45.0f),
                             (float)width/(float)height,
                             0.1f, 100.0f);
\end{lstlisting}

Esta opção dá um efeito de profundidade suficiente para perceber o jogo como 3D, mas sem complicar com uma câmara móvel (o que, neste tipo de jogo, raramente é essencial).

\section{Iluminação de Phong em \ac{GLSL}}
\label{chap2:sec:phong}

Para evitar um aspeto ``plano'', foi implementada uma iluminação do tipo Phong no \emph{fragment shader}. O modelo soma três componentes:
\begin{enumerate}
  \item \textbf{Ambiente}: iluminação constante (neste caso, $0.3$);
  \item \textbf{Difusa}: depende do ângulo entre a normal e a direção da luz;
  \item \textbf{Especular}: simula brilhos, dependente da posição do observador (expoente 32).
\end{enumerate}

O shader calcula o resultado final como:
\[
(\text{ambiente} + \text{difusa} + \text{especular}) \cdot \text{objectColor}
\]
o que permite reutilizar o mesmo shader para bola, raquete, blocos e paredes, variando apenas a cor do objeto.

Este modelo de iluminação é um dos mais usados em renderização em tempo real por ser simples, eficiente e suficientemente expressivo para objetos com materiais ``genéricos'' (plástico/metal), sendo descrito em manuais e referências comuns da área~\cite{realtimeRendering,learnopengl}.

\section{Conclusões}
\label{chap2:sec:concs}

As tecnologias escolhidas são adequadas ao objetivo do projeto: o par \ac{GLFW}+\ac{GLEW} resolve a parte de janela/contexto, o \ac{OpenGL} fornece um pipeline moderno e explícito, e o \ac{GLM} evita erros comuns de matemática. No capítulo seguinte (Capítulo~\ref{chap:arquitetura}) passamos da base tecnológica para a estrutura do jogo: requisitos, classes e o ciclo principal.

\clearpage{\thispagestyle{empty}\cleardoublepage}

\chapter{Análise e Arquitetura do Sistema}
\label{chap:arquitetura}

\section{Introdução}
\label{chap3:sec:intro}

Este capítulo descreve como o problema foi decomposto e traduzido para uma arquitetura simples, mas organizada. A implementação foi pensada para ser fácil de explicar numa defesa oral: existe uma classe central (\texttt{Game}) que controla estado, objetos e renderização; e classes pequenas para representar entidades do jogo (bola, raquete e blocos).

\section{Requisitos}
\label{chap3:sec:req}

A Tabela~\ref{tab:req} resume os requisitos definidos para o projeto.

\begin{table}[H]
\centering
\caption{Requisitos principais do projeto.}
\label{tab:req}
\begin{tabular}{p{2.2cm} p{11cm}}
\hline
\textbf{Tipo} & \textbf{Requisitos} \\
\hline
Funcionais &
(i) mover a raquete para esquerda/direita; 
(ii) atualizar bola com \textit{delta time};
(iii) colisões com paredes, raquete e blocos; 
(iv) destruir blocos e atualizar pontuação; 
(v) detetar vitória/derrota e permitir \textit{reset}. \\
\hline
Não funcionais &
(i) manter \ac{FPS} estável (sem cálculos pesados por frame);
(ii) código modular e legível;
(iii) uso do perfil \ac{OpenGL} 3.3 core e shaders em \ac{GLSL}. \\
\hline
\end{tabular}
\end{table}

\section{Estrutura de Classes}
\label{chap3:sec:classes}

O diagrama da Figura~\ref{fig:classes} mostra a organização em alto nível. A classe \texttt{Game} é responsável por:
\begin{itemize}
  \item Criar e inicializar os objetos do jogo;
  \item Gerir o input (\texttt{keys[1024]});
  \item Atualizar o estado a cada frame (\texttt{update(dt)});
  \item Desenhar a cena (\texttt{render()}).
\end{itemize}

As restantes classes são deliberadamente pequenas e com poucas responsabilidades: a bola sabe atualizar a sua posição e inverter componentes de velocidade; a raquete sabe mover-se respeitando limites; cada bloco sabe apenas se está destruído.

\begin{figure}[H]
\centering
\includegraphics[width=0.95\textwidth]{fig_classes_v4.pdf}
\caption{Visão geral das principais classes e relações (diagrama limpo).}
\label{fig:classes}
\end{figure}

\section{Ciclo Principal do Jogo}
\label{chap3:sec:loop}

O programa principal (em \texttt{main.cpp}) segue um ciclo típico de aplicações gráficas com \ac{GLFW}:
\begin{enumerate}
  \item calcular \texttt{deltaTime};
  \item processar input;
  \item atualizar estado;
  \item renderizar;
  \item trocar buffers e processar eventos.
\end{enumerate}

A Figura~\ref{fig:loop} sintetiza o fluxo por frame.

\begin{figure}[H]
\centering
\includegraphics[width=0.98\textwidth]{fig_game_loop_v4.pdf}
\caption{Ciclo principal executado em cada frame (com base no \texttt{main.cpp}).}
\label{fig:loop}
\end{figure}

Uma nota importante: o uso de \texttt{deltaTime} garante que a velocidade percebida se mantém aproximadamente constante, mesmo que a \ac{FPS} varie. Sem isto, o jogo ficaria mais rápido em máquinas mais potentes, o que é indesejável.

\section{Estados de Jogo e Regras}
\label{chap3:sec:estado}

A enumeração \texttt{GameState} define quatro estados: \texttt{GAME\_ACTIVE}, \texttt{GAME\_MENU}, \texttt{GAME\_WIN}, \texttt{GAME\_LOSE}. Nesta versão, o jogo inicia em \texttt{GAME\_ACTIVE} e muda para:
\begin{itemize}
  \item \textbf{Derrota} quando a bola passa a linha inferior (limite em $y=-10$);
  \item \textbf{Vitória} quando todos os blocos foram destruídos.
\end{itemize}

O estado \texttt{GAME\_MENU} está previsto na estrutura mas não é explorado (fica como ponto de extensão para trabalho futuro).

\section{Configuração do Mundo e Parâmetros}
\label{chap3:sec:param}

A lógica do jogo vive num ``mundo'' com limites simples, alinhados aos eixos (o que torna as colisões mais diretas). A Tabela~\ref{tab:param} agrega os parâmetros mais relevantes, extraídos diretamente do código.

\begin{table}[H]
\centering
\caption{Parâmetros principais do jogo (valores no código).}
\label{tab:param}
\begin{tabular}{p{5.0cm} p{3.5cm} p{4.5cm}}
\hline
\textbf{Parâmetro} & \textbf{Valor} & \textbf{Onde é usado} \\
\hline
Resolução & $1920\times1080$ & janela \ac{GLFW} (\texttt{main.cpp}) \\
Limites laterais & $x \in [-15,15]$ & paredes e limites da raquete/bola \\
Limite superior & $y=10$ & colisão da bola com ``teto'' \\
Linha de derrota & $y=-10$ & transição para \texttt{GAME\_LOSE} \\
Posição inicial da bola & $(0,-5,0)$ & inicialização e reset \\
Raquete: posição & $(0,-8,0)$ & início e reset \\
Raquete: tamanho & $(3,0.5,1)$ & escala do cubo \\
Raquete: velocidade & $20$ & movimento em $x$ \\
Bola: raio & $0.5$ & desenho e colisões \\
Grelha de blocos & $5\times10$ & \texttt{createBricks()} \\
Bloco: tamanho & $(2,1,1)$ & escala do cubo \\
Espaçamento entre blocos & $0.2$ & distribuição da grelha \\
Pontuação por bloco & $+10$ & destruição de blocos \\
\hline
\end{tabular}
\end{table}

\section{Controlos}
\label{chap3:sec:controls}

Os controlos são mapeados para duas teclas equivalentes em cada direção (A ou seta esquerda; D ou seta direita). Adicionalmente, existe uma tecla para reiniciar. Na prática, isto facilita testar em diferentes teclados e evita que o jogador dependa de um único esquema.

\begin{itemize}
  \item \textbf{A / $\leftarrow$}: mover raquete para a esquerda;
  \item \textbf{D / $\rightarrow$}: mover raquete para a direita;
  \item \textbf{R}: reiniciar o jogo;
  \item \textbf{ESC}: sair.
\end{itemize}

\section{Conclusões}
\label{chap3:sec:concs}

A arquitetura escolhida é intencionalmente simples, mas deixa espaço para evoluções (menu, níveis, diferentes tipos de blocos, etc.). No capítulo seguinte (Capítulo~\ref{chap:implementacao}) detalham-se as partes mais críticas da implementação: geração de geometria (cubo e esfera), shaders, colisões e desenho de todos os elementos em cena.

\clearpage{\thispagestyle{empty}\cleardoublepage}

\chapter{Implementação e Testes}
\label{chap:implementacao}

\section{Introdução}
\label{chap4:sec:intro}

Este capítulo descreve as principais decisões de implementação, com foco nas partes que, na prática, dão mais trabalho num projeto deste género: (i) criação de geometria e envio para a \ac{GPU}; (ii) shaders e iluminação; (iii) deteção de colisões e resposta; e (iv) testes e afinações.

Sempre que possível, são apresentados excertos de código curtos, retirados do projeto, para ligar a explicação diretamente ao que está a ser executado.

\section{Estrutura do Projeto}
\label{chap4:sec:estrutura}

A estrutura do projeto é simples e segue uma divisão comum entre interface (\texttt{include/}) e implementação (\texttt{src/}):

\begin{table}[H]
\centering
\caption{Estrutura principal do projeto.}
\label{tab:estrutura}
\begin{tabular}{p{3.0cm} p{11cm}}
\hline
\textbf{Diretório} & \textbf{Conteúdo} \\
\hline
\texttt{include/} & cabeçalhos: \texttt{game.h}, \texttt{ball.h}, \texttt{paddle.h}, \texttt{brick.h}, \texttt{renderer.h}, \texttt{shader.h} \\
\texttt{src/} & implementação das classes anteriores e \texttt{main.cpp} \\
\texttt{shaders/} & \texttt{vertex.vert} e \texttt{fragment.frag} (\ac{GLSL}) \\
\texttt{build/} & objetos compilados (\texttt{.o}) \\
\hline
\end{tabular}
\end{table}

\section{Documentação do Código (Doxygen)}
\label{chap4:sec:doxygen}

Para facilitar a leitura e a avaliação do projeto (especialmente quando o código cresce), foi seguida uma convenção de comentários compatível com \emph{Doxygen}~\cite{doxygenManual}. A ideia é simples: descrever, em poucas linhas, o objetivo de cada classe e das funções que têm mais lógica, sem encher o ficheiro de texto.

O excerto seguinte mostra o estilo de documentação usado (o conteúdo é representativo do que foi aplicado no projeto):

\begin{lstlisting}[caption={Exemplo de comentário Doxygen para uma função de colisões.},label={lst:doxygen}]
/**
 * @brief Testa colisão esfera--AABB (bola vs bloco).
 *
 * Calcula o ponto do bloco mais próximo do centro da bola (clamp) e compara
 * a distância com o raio. Se houver colisão, aplica uma reflexão em X ou Y.
 *
 * @param brick   Bloco a testar.
 * @return true se existe colisão, false caso contrário.
 */
bool Game::checkBallBrickCollision(Brick* brick);
\end{lstlisting}

Em ambiente de desenvolvimento, a documentação pode ser gerada com um ficheiro \texttt{Doxyfile} (incluído no projeto) e o comando:\\
\texttt{doxygen Doxyfile}

O resultado é uma página HTML com a descrição das classes, lista de métodos e ligações cruzadas, o que ajuda a navegar o código rapidamente.

\section{Gestão do Projeto}
\label{chap4:sec:gestao}

Embora o projeto seja relativamente pequeno, optou-se por uma abordagem incremental ("primeiro pôr a renderizar, depois pôr a jogar"). Na prática, isto evitou o erro comum de escrever muita lógica antes de existir um ciclo de renderização estável para testar.

O desenvolvimento foi dividido em etapas curtas, cada uma com um resultado visível e testável:

\begin{table}[H]
\centering
\caption{Plano de trabalho (etapas) e objetivo de cada etapa.}
\label{tab:plano}
\begin{tabular}{p{3.6cm} p{10.4cm}}
\hline
\textbf{Etapa} & \textbf{Objetivo} \\
\hline
Setup base & Janela + contexto \ac{OpenGL} 3.3, “game loop” com \texttt{deltaTime} e limpeza de buffers. \\
Shaders & Carregar/compilar shaders e desenhar um objeto simples para validar o pipeline. \\
Geometria & Criar cubo (raquete/blocos/parede) e esfera (bola) com \ac{VAO}/\ac{VBO}. \\
Entidades & Implementar classes \texttt{Ball}, \texttt{Paddle} e \texttt{Brick} e integrá-las em \texttt{Game}. \\
Colisões & Colisão esfera--AABB (blocos/raquete) + colisões com paredes, score e estados. \\
Afinações & Ajustes de velocidades, limites do mundo, e “spin” para a jogabilidade ficar controlável. \\
\hline
\end{tabular}
\end{table}

Este tipo de divisão ajudou a isolar problemas: por exemplo, quando algo corria mal nas colisões, já existia renderização e input a funcionar, o que facilitou repetir cenários e comparar.

\section{Descrição do Código (classes e funções)}
\label{chap4:sec:codigo}

O código foi organizado para que cada classe tenha uma responsabilidade clara. A Tabela~\ref{tab:classes-api} resume a “API” mais relevante de cada módulo.

\begin{table}[H]
\centering
\caption{Principais classes e métodos (visão orientada à defesa).}
\label{tab:classes-api}
\begin{tabular}{p{2.6cm} p{6.7cm} p{4.7cm}}
\hline
\textbf{Classe} & \textbf{Responsabilidade} & \textbf{Métodos-chave} \\
\hline
\texttt{Game} & Orquestra o ciclo do jogo, cria entidades, gere estado, score e renderização. & \texttt{init()}, \texttt{processInput(dt)}, \texttt{update(dt)}, \texttt{render()}, \texttt{reset()} \\
\texttt{Ball} & Estado e movimento da bola (esfera) e pequenas regras (inversões/spin). & \texttt{update(dt)}, \texttt{reverseX/Y()}, \texttt{addSpin()} \\
\texttt{Paddle} & Movimento controlado da raquete com limites no eixo X. & \texttt{moveLeft()}, \texttt{moveRight()}, \texttt{reset()} \\
\texttt{Brick} & Representa um bloco e o seu estado (“destroyed”). & \texttt{destroy()}, \texttt{reset()} \\
\texttt{Renderer} & Cria e guarda geometria em buffers (cubo + esfera). & \texttt{init()}, \texttt{cleanup()}, \texttt{createCube()}, \texttt{createSphere(...)} \\
\texttt{Shader} & Carregamento/compilação/link de shaders + setters de uniforms. & \texttt{use()}, \texttt{setMat4()}, \texttt{setVec3()}, \texttt{checkCompileErrors()} \\
\hline
\end{tabular}
\end{table}

Para além da estrutura, há dois pontos que são muito úteis de mostrar numa defesa: (i) como é feito o \textbf{mapeamento de teclas} para movimento e reset; e (ii) como se implementam rapidamente os \textbf{limites do mundo} (paredes e fundo).

O processamento de input é feito através de um vetor \texttt{keys[1024]} atualizado por um \emph{callback} do \ac{GLFW}, e consumido por \texttt{Game::processInput(dt)}:

\begin{lstlisting}[caption={Processamento de input e limites laterais da raquete.},label={lst:processInput}]
void Game::processInput(float dt) {
    if (state == GAME_ACTIVE) {
        if (keys[GLFW_KEY_A] || keys[GLFW_KEY_LEFT]) {
            paddle->moveLeft(dt, -15.0f);
        }
        if (keys[GLFW_KEY_D] || keys[GLFW_KEY_RIGHT]) {
            paddle->moveRight(dt, 15.0f);
        }
    }

    if (keys[GLFW_KEY_R]) {
        reset();
    }
}
\end{lstlisting}

Os limites do mundo são tratados de forma direta no \texttt{update(dt)}: ao tocar nas paredes laterais ou no teto, a bola inverte a componente correspondente; ao ultrapassar o fundo (linha de derrota), o estado passa para \texttt{GAME\_LOSE}.

\begin{lstlisting}[caption={Colisões da bola com paredes, teto e fundo (game over).},label={lst:wallCollisions}]
if (ball->position.x - ball->radius <= -15.0f ||
    ball->position.x + ball->radius >= 15.0f) {
    ball->reverseX();
}
if (ball->position.y + ball->radius >= 10.0f) {
    ball->reverseY();
}
if (ball->position.y - ball->radius <= -10.0f) {
    state = GAME_LOSE;
}
\end{lstlisting}

\section{Inicialização da Janela e do Contexto OpenGL}
\label{chap4:sec:init}

A inicialização é feita em \texttt{main.cpp}. Os pontos mais importantes são:
\begin{itemize}
  \item criação da janela com \ac{GLFW} e definição da versão \ac{OpenGL} 3.3 core;
  \item inicialização do \ac{GLEW};
  \item ativação do teste de profundidade (\texttt{glEnable(GL\_DEPTH\_TEST)}), necessário para objetos 3D.
\end{itemize}

Para ligar a explicação ao código, o excerto seguinte mostra o coração do \emph{game loop}, com cálculo de \textit{delta time} e chamadas às três fases clássicas: \emph{input} $\rightarrow$ \emph{update} $\rightarrow$ \emph{render}.

\begin{lstlisting}[caption={Excerto do \emph{game loop} em \texttt{main.cpp} com \textit{delta time}.},label={lst:gameloop}]
while (!glfwWindowShouldClose(window)) {
    float currentFrame = glfwGetTime();
    deltaTime = currentFrame - lastFrame;
    lastFrame = currentFrame;

    breakout->processInput(deltaTime);
    breakout->update(deltaTime);

    glClearColor(0.1f, 0.1f, 0.15f, 1.0f);
    glClear(GL_COLOR_BUFFER_BIT | GL_DEPTH_BUFFER_BIT);
    breakout->render();

    glfwSwapBuffers(window);
    glfwPollEvents();
}
\end{lstlisting}

\section{Renderer: Cubo e Esfera}
\label{chap4:sec:renderer}

A classe \texttt{Renderer} encapsula a criação de buffers e \ac{VAO} para duas malhas:
\begin{itemize}
  \item um \textbf{cubo} (usado para raquete, blocos e paredes);
  \item uma \textbf{esfera} (usada para a bola).
\end{itemize}

O cubo é definido por 36 vértices (12 triângulos), cada um com posição e normal. A esfera é gerada de forma paramétrica com \texttt{stacks} e \texttt{sectors} (neste caso, 16 e 32), produzindo triângulos a partir de uma grelha em coordenadas esféricas. Esta abordagem é comum porque é rápida, não exige ficheiros externos e é suficiente para uma bola com sombreamento.

\section{Shaders e Uniforms}
\label{chap4:sec:shaders}

O shader é carregado a partir de ficheiros (\texttt{shaders/vertex.vert} e \texttt{shaders/fragment.frag}). O programa compila e faz \emph{link} e imprime mensagens de erro caso algo falhe (útil na fase de desenvolvimento).

No \texttt{render()}, a classe \texttt{Game} atualiza sempre os seguintes \emph{uniforms}:
\begin{itemize}
  \item matrizes \texttt{model}, \texttt{view}, \texttt{projection};
  \item posição da luz (\texttt{lightPos});
  \item posição da câmara (\texttt{viewPos});
  \item cor do objeto (\texttt{objectColor}).
\end{itemize}

Isto permite desenhar toda a cena com um único shader, variando apenas o modelo (transformação) e a cor.

\section{Deteção de Colisão: Esfera vs. \ac{AABB}}
\label{chap4:sec:colisoes}

A colisão mais interessante (e mais ``propensa a bugs'') é entre a bola (esfera) e os blocos (caixas alinhadas aos eixos). A técnica usada é a padrão: calcular o ponto mais próximo da esfera dentro do AABB através de \textbf{clamp} e comparar distâncias.

\begin{lstlisting}[caption={Colisão esfera--AABB (bola vs bloco).},label={lst:sphereAABB}]
glm::vec3 difference = center - aabb_center;
glm::vec3 clamped = glm::clamp(difference, -aabb_half, aabb_half);
glm::vec3 closest = aabb_center + clamped;
difference = closest - center;

if (glm::length(difference) < ball->radius) {
    glm::vec3 absD = glm::abs(difference);
    if (absD.y > absD.x) ball->reverseY();
    else                 ball->reverseX();
    return true;
}
\end{lstlisting}

A leitura do algoritmo é a seguinte:
\begin{enumerate}
  \item passa-se para o referencial do bloco (\texttt{difference});
  \item limita-se cada componente ao intervalo do AABB (\texttt{clamped});
  \item obtém-se o ponto do AABB mais próximo do centro da esfera (\texttt{closest});
  \item se a distância for menor que o raio, existe colisão;
  \item a resposta é uma reflexão em $x$ ou $y$ consoante a direção dominante da penetração.
\end{enumerate}

Esta resposta é simples e funciona bem para um \emph{Breakout}. No entanto, é importante notar que ela pode falhar em colisões ``de canto'' muito rápidas (problema conhecido como \emph{tunneling}). No trabalho futuro (Capítulo~\ref{chap:conclusoes}) sugere-se como melhorar este ponto.

\section{Colisão Bola--Raquete e Spin}
\label{chap4:sec:paddle}

A colisão com a raquete usa uma verificação de sobreposição em X/Y/Z (também uma AABB, mas com a bola como esfera). Existe ainda uma condição adicional: só conta como colisão se a bola estiver a descer (\texttt{velocity.y < 0}), para evitar múltiplas inversões quando a bola já está a subir.

\begin{lstlisting}[caption={Teste de colisão bola--raquete.},label={lst:ballPaddle}]
return collisionX && collisionY && collisionZ && ball->velocity.y < 0;
\end{lstlisting}

Quando a colisão acontece, a bola inverte $y$ e a posição é ajustada para ficar acima da raquete (evita que fique ``presa'' dentro do cubo). Depois aplica-se um \textbf{spin} baseado no ponto de contacto:

\begin{lstlisting}[caption={Spin: influencia horizontal em função do ponto de contacto.},label={lst:spin}]
float hitPos = (ball->position.x - paddle->position.x) / (paddle->size.x / 2.0f);
ball->addSpin(hitPos * 0.02f);
\end{lstlisting}

Na bola, o spin é escalado e limitado para manter o jogo controlável:

\begin{lstlisting}[caption={Aplicação e limite do spin na bola.},label={lst:spinBall}]
velocity.x += spin * 10.0f;
if (velocity.x > 12.0f)  velocity.x = 12.0f;
if (velocity.x < -12.0f) velocity.x = -12.0f;
\end{lstlisting}

Esta combinação dá ao jogador alguma influência sobre a trajetória sem tornar a bola ``imprevisível''.

\section{Renderização da Cena}
\label{chap4:sec:render}

A cena é desenhada sempre na mesma ordem:
\begin{enumerate}
  \item raquete (cubo escalado);
  \item bola (esfera escalada ao raio);
  \item blocos (cubos escalados ao tamanho do bloco);
  \item paredes laterais, teto e linha inferior (cubos finos/largos).
\end{enumerate}

Para as paredes, o mesmo VAO do cubo é reutilizado e apenas se altera a matriz \texttt{model}. A ``linha de perigo'' inferior é desenhada a vermelho, o que ajuda a dar feedback imediato ao jogador sobre onde perde o jogo.


\section{Aspeto Visual e Captura de Ecrã}
\label{chap4:sec:screenshot}

A Figura~\ref{fig:screenshot} apresenta uma captura de ecrã do jogo em execução. 
Esta imagem é útil para contextualizar rapidamente o resultado final: grelha de blocos por linhas de cor, 
raquete na zona inferior e bola com sombreamento, já com iluminação ativa e teste de profundidade.

\begin{figure}[H]
  \centering
  \includegraphics[width=0.95\textwidth]{Breakout.png}
  \caption{Captura de ecrã do \emph{3D Breakout} em execução.}
  \label{fig:screenshot}
\end{figure}


\section{Testes Realizados}
\label{chap4:sec:testes}

Não foi criada uma suite automática, pelo que os testes foram maioritariamente manuais e focados em cenários que normalmente geram problemas em jogos de colisões simples. A Tabela~\ref{tab:testes} resume os cenários testados.

\begin{table}[H]
\centering
\caption{Testes práticos realizados.}
\label{tab:testes}
\begin{tabular}{p{4.2cm} p{6.2cm} p{3.6cm}}
\hline
\textbf{Cenário} & \textbf{O que se valida} & \textbf{Resultado} \\
\hline
Colisão com paredes ($x=\pm 15$) & inversão correta de $v_x$ sem atravessar a parede & OK \\
Colisão com teto ($y=10$) & inversão de $v_y$ e continuidade do movimento & OK \\
Colisão com raquete & inversão apenas quando bola desce + reposicionamento & OK \\
Spin na raquete & variação de $v_x$ conforme ponto de contacto & OK \\
Colisão com bloco (centro) & destruição do bloco + pontuação + inversão coerente & OK \\
Colisão com bloco (canto) & avaliar reflexões em $x$ vs $y$ & Aceitável \\
Derrota ($y\le -10$) & transição para \texttt{GAME\_LOSE} + mensagem & OK \\
Vitória & quando todos os blocos destruídos & OK \\
Reset (tecla R) & reposicionar objetos e restaurar blocos/pontos & OK\footnotemark \\
\hline
\end{tabular}
\end{table}
\footnotetext{No \textit{reset}, a velocidade definida no código (\texttt{(0.05, 0.05, 0)}) é muito baixa e pode dar a sensação de ``bola parada''. Uma melhoria simples é restaurar a velocidade inicial usada na inicialização.}

\section{Observações de Desempenho}
\label{chap4:sec:perf}

O custo por frame é baixo: existem poucas \emph{draw calls} (raquete + bola + até 50 blocos + paredes) e a geometria é simples. A esfera é a malha mais pesada, mas continua dentro de valores confortáveis para hardware atual. Como a atualização física é básica (operações vetoriais e poucas comparações), o gargalo do projeto tende a ser a renderização e não a lógica.

\section{Conclusões}
\label{chap4:sec:concs}

A implementação cumpre o objetivo do projeto: jogo funcional, com colisões estáveis, iluminação e uma estrutura clara. Ao mesmo tempo, ficaram identificados pontos de melhoria realistas (por exemplo, corrigir a velocidade do reset e refinar colisões em cantos rápidos), discutidos no Capítulo~\ref{chap:conclusoes}.

\clearpage{\thispagestyle{empty}\cleardoublepage}

\chapter{Conclusões e Trabalho Futuro}
\label{chap:conclusoes}

\section{Conclusões}
\label{sec:conc}

O projeto \emph{3D Breakout} atingiu os objetivos propostos: existe um jogo jogável em 3D, com controlo responsivo da raquete, colisões com paredes/raquete/blocos, pontuação e deteção de vitória/derrota. Do ponto de vista de computação gráfica, a aplicação utiliza um pipeline moderno de \ac{OpenGL} com shaders em \ac{GLSL}, suportado por \ac{GLFW}, \ac{GLEW} e \ac{GLM}.

Um aspeto que considero particularmente positivo é a separação de responsabilidades: 
\begin{itemize}
  \item \texttt{Game} funciona como ponto central do estado e da lógica;
  \item \texttt{Renderer} trata da geometria e buffers;
  \item \texttt{Shader} encapsula carregamento/compilação e \emph{uniforms};
  \item \texttt{Ball}, \texttt{Paddle} e \texttt{Brick} mantêm o modelo de dados simples e fácil de testar.
\end{itemize}

Ao longo do desenvolvimento, o maior esforço esteve na parte das colisões e na ``sensação'' do jogo (velocidades e spin). Mesmo com uma física simplificada, pequenos detalhes (como reposicionar a bola após colisão com a raquete) têm impacto direto na estabilidade e no controlo.

\section{Limitações Identificadas}
\label{sec:limit}

As principais limitações são:

\begin{itemize}
  \item \textbf{Resposta de colisão simplificada}: a bola apenas reflete em $x$ ou $y$ consoante a componente dominante, o que pode gerar casos pouco realistas em cantos.
  \item \textbf{Possível \emph{tunneling}}: com velocidades mais altas, a bola pode atravessar um bloco entre frames. Este é um problema comum quando se usa deteção discreta (por frame) em vez de contínua.
  \item \textbf{Reset com velocidade muito baixa}: no código atual, o reset define a velocidade para \texttt{(0.05, 0.05, 0)}, o que pode tornar o reinício pouco dinâmico sem nova afinação.
  \item \textbf{Ausência de HUD}: a pontuação e estado são impressos na consola; não existe renderização de texto na janela.
\end{itemize}

\section{Trabalho Futuro}
\label{sec:fut}

Existem várias extensões possíveis que melhorariam tanto a qualidade do jogo como o valor técnico do projeto:

\begin{enumerate}
  \item \textbf{HUD e menus}: renderização de texto (por exemplo, com uma biblioteca de \emph{font rendering}) para mostrar pontuação, estado e instruções no ecrã, e implementação real do estado \texttt{GAME\_MENU}.
  \item \textbf{Vidas e níveis}: introduzir vidas, aumentar progressivamente a dificuldade e criar múltiplos níveis (várias grelhas de blocos).
  \item \textbf{Melhorias de física}: normalizar velocidade da bola, introduzir aceleração gradual e melhorar a resposta de colisão para considerar o vetor normal de impacto de forma mais robusta.
  \item \textbf{Colisão contínua}: implementar \emph{swept sphere} contra AABB para reduzir \emph{tunneling} e tornar o comportamento mais consistente a altas velocidades.
  \item \textbf{Texturas e materiais}: adicionar texturas ou diferentes materiais para distinguir bola/raquete/blocos e enriquecer o aspeto visual.
  \item \textbf{Som}: efeitos sonoros em colisões e vitória/derrota para aumentar o feedback ao jogador.
\end{enumerate}

\section{Consideração Final}
\label{sec:final}

Mesmo sendo um projeto relativamente curto, o desenvolvimento ajudou a consolidar os conceitos mais relevantes de computação gráfica em tempo real. Em particular, ficou claro que a parte ``gráfica'' (shaders e matrizes) e a parte ``de jogo'' (colisões e afinação) estão sempre ligadas: um pequeno erro num vetor ou num limite do mundo traduz-se imediatamente em comportamento estranho no ecrã. É precisamente este ciclo rápido de testar/ajustar que torna este tipo de projeto tão útil num contexto académico.

\clearpage{\thispagestyle{empty}\cleardoublepage}

\appendix
\chapter{Manual de Utilização}
\label{ap:A}

\section{Como Jogar}

O objetivo é destruir todos os blocos no topo do campo, fazendo a bola ricochetear na raquete. Sempre que um bloco é destruído, a pontuação aumenta. O jogo termina em \textbf{vitória} quando todos os blocos estão destruídos, ou em \textbf{derrota} quando a bola passa a linha inferior (a linha vermelha).

\section{Controlos}

\begin{table}[H]
\centering
\caption{Controlos do jogo.}
\label{tab:controls}
\begin{tabular}{p{4.5cm} p{9.5cm}}
\hline
\textbf{Tecla} & \textbf{Ação} \\
\hline
A ou $\leftarrow$ & mover a raquete para a esquerda \\
D ou $\rightarrow$ & mover a raquete para a direita \\
R & reiniciar (reset) o jogo \\
ESC & sair \\
\hline
\end{tabular}
\end{table}

\section{Notas de Jogabilidade}

\begin{itemize}
  \item A trajetória da bola pode ser influenciada pelo ponto onde ela bate na raquete (``spin''): bater mais à esquerda tende a enviar a bola para a esquerda e vice-versa.
  \item A linha vermelha na parte inferior serve de referência visual para a zona de derrota.
\end{itemize}

\clearpage{\thispagestyle{empty}\cleardoublepage}
\chapter{Compilação, Execução e Estrutura}
\label{ap:B}

\section{Dependências}

Para compilar o projeto são necessárias as seguintes dependências:

\begin{itemize}
  \item Compilador com suporte a C++17 (ex.: \texttt{g++}).
  \item Bibliotecas: \ac{GLFW}, \ac{GLEW}, \ac{OpenGL} e \ac{GLM}.
\end{itemize}

\section{Compilação e Execução (Windows)}

O \texttt{Makefile} fornecido está orientado para Windows, usando \texttt{g++} (por exemplo, via MinGW) e ligando com:

\begin{itemize}
  \item \texttt{-lglfw3}
  \item \texttt{-lglew32}
  \item \texttt{-lopengl32}
  \item \texttt{-lgdi32}
\end{itemize}

Com as dependências instaladas, basta executar:

\begin{lstlisting}[caption={Compilar e executar.},label={lst:buildrun}]
make
make run
\end{lstlisting}

Para limpar ficheiros intermédios:

\begin{lstlisting}[caption={Limpar build.},label={lst:clean}]
make clean
\end{lstlisting}

\section{Compilação (Linux/macOS)}

Em Linux, tipicamente substitui-se o conjunto de bibliotecas por algo do género:

\begin{itemize}
  \item \texttt{-lglfw -lGLEW -lGL}
\end{itemize}

Em macOS (via Homebrew), poderá ser necessário ajustar caminhos de includes e libs e usar \texttt{-framework OpenGL}. Como isto depende bastante da máquina, a recomendação é manter o código igual e apenas adaptar os \textit{flags} de ligação.

\section{Estrutura de Ficheiros}

\begin{lstlisting}[caption={Estrutura do projeto (simplificada).},label={lst:tree}]
3D-Breakout/
  include/
    ball.h  brick.h  game.h  paddle.h  renderer.h  shader.h
  src/
    ball.cpp brick.cpp game.cpp main.cpp paddle.cpp renderer.cpp shader.cpp
  shaders/
    vertex.vert fragment.frag
  Makefile
\end{lstlisting}

\section{Notas de Depuração}

Durante o desenvolvimento, foi útil:
\begin{itemize}
  \item imprimir o estado e pontuação na consola (vitória/derrota);
  \item reduzir temporariamente a velocidade da bola para observar colisões;
  \item desenhar paredes/limites para perceber melhor a área de jogo.
\end{itemize}

\section{Compilação do Relatório (\LaTeX)}

Para compilar este relatório (incluindo a bibliografia), recomenda-se a sequência:

\begin{lstlisting}[caption={Compilação do relatório em \LaTeX.},label={lst:buildreport}]
pdflatex relatorio-projeto.tex
biber relatorio-projeto
pdflatex relatorio-projeto.tex
pdflatex relatorio-projeto.tex
\end{lstlisting}

Esta repetição é normal: garante que o índice, a lista de figuras/tabelas e as referências cruzadas ficam preenchidas.


\clearpage{\thispagestyle{empty}\cleardoublepage}

\backmatter

\printbibliography[heading=bibintoc,title={Bibliografia}]

\end{document}

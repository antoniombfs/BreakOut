\chapter{Conclusões e Trabalho Futuro}
\label{chap:conclusoes}

\section{Conclusões}
\label{sec:conc}

O projeto \emph{3D Breakout} atingiu os objetivos propostos: existe um jogo jogável em 3D, com controlo responsivo da raquete, colisões com paredes/raquete/blocos, pontuação e deteção de vitória/derrota. Do ponto de vista de computação gráfica, a aplicação utiliza um pipeline moderno de \ac{OpenGL} com shaders em \ac{GLSL}, suportado por \ac{GLFW}, \ac{GLEW} e \ac{GLM}.

Um aspeto que considero particularmente positivo é a separação de responsabilidades: 
\begin{itemize}
  \item \texttt{Game} funciona como ponto central do estado e da lógica;
  \item \texttt{Renderer} trata da geometria e buffers;
  \item \texttt{Shader} encapsula carregamento/compilação e \emph{uniforms};
  \item \texttt{Ball}, \texttt{Paddle} e \texttt{Brick} mantêm o modelo de dados simples e fácil de testar.
\end{itemize}

Ao longo do desenvolvimento, o maior esforço esteve na parte das colisões e na ``sensação'' do jogo (velocidades e spin). Mesmo com uma física simplificada, pequenos detalhes (como reposicionar a bola após colisão com a raquete) têm impacto direto na estabilidade e no controlo.

\section{Limitações Identificadas}
\label{sec:limit}

As principais limitações são:

\begin{itemize}
  \item \textbf{Resposta de colisão simplificada}: a bola apenas reflete em $x$ ou $y$ consoante a componente dominante, o que pode gerar casos pouco realistas em cantos.
  \item \textbf{Possível \emph{tunneling}}: com velocidades mais altas, a bola pode atravessar um bloco entre frames. Este é um problema comum quando se usa deteção discreta (por frame) em vez de contínua.
  \item \textbf{Reset com velocidade muito baixa}: no código atual, o reset define a velocidade para \texttt{(0.05, 0.05, 0)}, o que pode tornar o reinício pouco dinâmico sem nova afinação.
  \item \textbf{Ausência de HUD}: a pontuação e estado são impressos na consola; não existe renderização de texto na janela.
\end{itemize}

\section{Trabalho Futuro}
\label{sec:fut}

Existem várias extensões possíveis que melhorariam tanto a qualidade do jogo como o valor técnico do projeto:

\begin{enumerate}
  \item \textbf{HUD e menus}: renderização de texto (por exemplo, com uma biblioteca de \emph{font rendering}) para mostrar pontuação, estado e instruções no ecrã, e implementação real do estado \texttt{GAME\_MENU}.
  \item \textbf{Vidas e níveis}: introduzir vidas, aumentar progressivamente a dificuldade e criar múltiplos níveis (várias grelhas de blocos).
  \item \textbf{Melhorias de física}: normalizar velocidade da bola, introduzir aceleração gradual e melhorar a resposta de colisão para considerar o vetor normal de impacto de forma mais robusta.
  \item \textbf{Colisão contínua}: implementar \emph{swept sphere} contra AABB para reduzir \emph{tunneling} e tornar o comportamento mais consistente a altas velocidades.
  \item \textbf{Texturas e materiais}: adicionar texturas ou diferentes materiais para distinguir bola/raquete/blocos e enriquecer o aspeto visual.
  \item \textbf{Som}: efeitos sonoros em colisões e vitória/derrota para aumentar o feedback ao jogador.
\end{enumerate}

\section{Consideração Final}
\label{sec:final}

Mesmo sendo um projeto relativamente curto, o desenvolvimento ajudou a consolidar os conceitos mais relevantes de computação gráfica em tempo real. Em particular, ficou claro que a parte ``gráfica'' (shaders e matrizes) e a parte ``de jogo'' (colisões e afinação) estão sempre ligadas: um pequeno erro num vetor ou num limite do mundo traduz-se imediatamente em comportamento estranho no ecrã. É precisamente este ciclo rápido de testar/ajustar que torna este tipo de projeto tão útil num contexto académico.
